\section{Introduction}
We propose a system to allow users to provide both negative and positive image queries in order to retrieve images with similar content.
One important application of our system is to allow users to search through their personal photo collections with content-based queries, and as such the
system should not sacrifice performance at smaller scales in order to achieve scalability.
We plan to apply modern machine learning techniques in the context of computer vision to compactly express the content and features of the images, then leverage
this representation to provide efficient querying of the data set.

Evaluation of content-based image retrieval is qualitative, making it difficult to evaluate the success of this system.
As such, the evaluation will be closely tied to qualitative analysis through user studies.
Performance will be evaluated against other state of the art content-based image retrieval systems.

Content-based image retrieval allows retrieval of multimedia data based on the content of the image as opposed to metadata such as text annotations.
In many cases, such annotations are absent or incomplete, necessitating this content-based approach to improve the accuracy and completeness of search results.


%Note: talk about how our system hopes to use a much simpler user interface than approaches like AverageExplorer which are quite complex and poorly suited for mobile.
%Each image in its entirety will be a positive or negative query

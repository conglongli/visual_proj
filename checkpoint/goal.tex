\section{Resources}
We will be building our system on top of code taken from Lin et al.~\cite{gitrepo} from their CVPR 2015 workshop paper.
For hardware, we will be using the maas nodes owned by CMCL (maas.cmcl.cs.cmu.edu).
These machines are equipped with GTX 980 graphics cards, and there are 10 available for use.

We plan on using a subset of Yahoo Webscope's 14.5tb image dataset to evaluate our system.
To prototype the system we will be using the CIFAR-10 and MNIST datasets.
CIFAR-10 contains 60,000 images, which we believe is large enough to provide a baseline for development
without incurring long runtimes during prototyping.

\section{Evaluation Plan and Goals}

Our first goal is to take the work done by Lin et al. on representing images as binary strings based on neural network activations,
and build a system that takes a simple brute force approach to evaluating similarity by L1 or L2 euclidean distance between the query and the dataset images.
We hope to make this simple system reasonably fast by leveraging the inherently parallel nature of this workload.

Our next step is to explore different methods of calculating similarity between images.
In particular, we hope to use a 2-3 layer CNN consisting of a ReLU layer sandwiched between fully connected layers,
training briefly on the set of input images then classifying images in the database using this CNN.
We hope that this approach yields higher accuracy.
Running a neural network will be a fairly expensive operation compared to a simple L1/L2 euclidean distance calculation,
so it is likely we will need to modify our approach to making this efficient.
At the moment it is not clear what this will entail, but our hope is that it will become apparent after analyzing
the performance characteristics of the system.

Our final stretch goal is to scale this system to larger data sets on the order of multiple terabytes,
modifying it to be performant on these larger data sets.
To achieve this goal, we will run our system on a larger dataset and work to determine and alleviate performance bottlenecks.

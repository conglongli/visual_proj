\section{Introduction}
Existing content-based image retrieval (CBIR) systems allow users to search for similar images
for a single image. Recent CBIR systems improves the quality of results by
using the Euclidean distances between feature
vectors to find the similar images with similar features. On the other hand, users
may like/dislike some of the returned similar images, but it's difficult to send this
feedback to the server under the current system.

To the end of this, we propose a CBIR system to allow users to provide multiple negative and positive image queries in order to retrieve images with similar content.
With such CBIR system, users can repeatedly send refined queries until satisfied with
the return similar images.
One important application of our system is to allow users to search through their personal photo collections with content-based queries, and as
such we will target datasets consisting of a reasonable number of images for a personal collection (1,000,000 images/~1tb).
We implement two different algorithms to find similar images for multiple query images, and
use different techniques to improve the performance of the processing.

We evaluate the CBIR system both qualitatively and quantitatively. In qualitative evaluation, our
system constantly produce high-quality similar images of the query images. In quantitative
evaluation, our system is able to process a 10-image query in 4 seconds when using a 300,000-image
dataset.

% Evaluation of content-based image retrieval is qualitative, making it difficult to evaluate the success of this system.
% As such, the evaluation will be closely tied to qualitative analysis through user studies.
% Performance will be evaluated against other state of the art content-based image retrieval systems.

% Content-based image retrieval allows retrieval of multimedia data based on the content of the image as opposed to metadata such as text annotations.
% In many cases, such annotations are absent or incomplete, necessitating this content-based approach to improve the accuracy and completeness of search results.

%Note: talk about how our system hopes to use a much simpler user interface than approaches like AverageExplorer which are quite complex and poorly suited for mobile.
%Each image in its entirety will be a positive or negative query

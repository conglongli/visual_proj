\section{Related Work}
\subsection{Content-Based Image Retrieval}
Content-based image retrieval aims at searching for similar images
through the analysis of image content. As DNNs learn rich mid-level
image descriptors, ImageNet uses the feature vectors from the 7th
layer in image retrieval and demonstrated outstanding
performance~\cite{krizhevsky2012imagenet}. Babenko \textit{et al.}
proposed to compress the DNN features using PAC and discriminative
dimensionality reduction to improve the efficiency~\cite{babenko2014neural}.

Due to the recent growth of visual contents, rapid search in a large
database becomes an emerging need. In stead of linear search which has
high computational cost, a practical strategy is to use the technique
of Approximate Nearest Neighbor (ANN) or hashing based method for
speed up~\cite{gionis1999similarity,weiss2009spectral,kulis2009learning,
norouzi2011minimal,liu2012supervised,xia2014supervised}. These methods
project the high-dimensional features to a lower dimensional space, and
then generate the compact binary codes. Benefiting from the produced
binary codes, fast image search can be carried out via binary pattern
matching or Hamming distance measurement. However, these methods require
to use similarity matrix to describe the relationship of the image pairs, which
is no practical for a large-scale dataset. Recently Lin \textit{et al.}
propose an effective deep learning framework to generate binary hash codes
and image representations in a point-wised manner, making it suitable
for large-scale datasets~\cite{lin2015deep}.

\subsection{Image Analysis Applications}
AverageExplorer is an interactive framework that allows an user to rapidly
explore and visualize a large image collection using the medium of average
images~\cite{zhu2014averageexplorer}. This real-time system provides a
way to summarize large amounts of visual data by weighted averages of
an image collection, with the weights reflecting user-indicated importance.
Zhu \textit{et al.} propose a way to help user-controlled realistic image
manipulation by learning the manifold of natural images and defining the
image editing operations with constraints lie on the learned manifold at
all times. Thus the model automatically adjusts the output keeping all edits
as realistic as possible~\cite{zhu2016generative}.

\section{Algorithm}